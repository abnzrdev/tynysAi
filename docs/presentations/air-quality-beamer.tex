\documentclass{beamer}

\usetheme{Madrid}
\usecolortheme{default}

\title{Low-Cost IoT Platform for In-Cabin Air Quality Monitoring}
\subtitle{Pilot Deployment and Machine Learning Feasibility Study}
\author{Abenezer Alemayehu}
\institute{Almaty, Kazakhstan}
\date{\today}

\begin{document}

% Title Slide
\begin{frame}
  \titlepage
\end{frame}

% Problem Statement
\begin{frame}{Why In-Cabin Air Quality Matters}
\begin{itemize}
  \item Poor air quality affects health, comfort, and concentration
  \item Vehicles and closed spaces accumulate pollutants quickly
  \item Professional monitoring systems are expensive and static
\end{itemize}
\end{frame}

% Project Overview
\begin{frame}{Project Overview}
\begin{itemize}
  \item Low-cost IoT device for real-time air quality monitoring
  \item Live web platform for visualization
  \item Machine learning used for automatic air quality classification
  \item Pilot case study conducted in Almaty
\end{itemize}
\end{frame}

% System Architecture
\begin{frame}{System Architecture}
\begin{itemize}
  \item IoT sensor device (Tynys)
  \item Cloud database
  \item Web-based dashboard
  \item Machine learning inference layer
\end{itemize}
\end{frame}

% Measured Parameters
\begin{frame}{Measured Parameters}
\begin{columns}
\column{0.5\textwidth}
\begin{itemize}
  \item CO$_2$
  \item PM$_{2.5}$
  \item PM$_{10}$
  \item PM$_1$
  \item Temperature
  \item Humidity
\end{itemize}

\column{0.5\textwidth}
\begin{itemize}
  \item VOC
  \item CO
  \item O$_3$
  \item NO$_2$
  \item CH$_2$O
\end{itemize}
\end{columns}
\end{frame}

% Data Format
\begin{frame}[fragile]{Data Transmission Format}
Sensor data is sent in JSON format:
\begin{verbatim}
{
 "device_id": "lab01",
 "site": "AGI_Lab",
 "pm25": 25.7,
 "pm10": 43.1,
 "co2": 412,
 "temp": 21.8,
 "hum": 46.2
}
\end{verbatim}
\end{frame}

% Sampling and Filtering
\begin{frame}{Sampling and Noise Reduction}
\begin{itemize}
  \item Data sampled every 60 seconds
  \item 5-point moving average filter applied
\end{itemize}

\[
  y = \frac{x_1 + x_2 + x_3 + x_4 + x_5}{5}
\]

\end{frame}

% Machine Learning
\begin{frame}{Machine Learning Approach}
The system automatically classifies air quality levels:
\begin{itemize}
  \item Low / Moderate / High
  \item Safe / Unsafe
\end{itemize}
\end{frame}

% Algorithms
\begin{frame}{Algorithms Used}
\begin{itemize}
  \item Logistic Regression
  \item Decision Tree
  \item Random Forest
  \item XGBoost (main model)
\end{itemize}
\end{frame}

% Calibration
\begin{frame}{Sensor Calibration}
\begin{itemize}
  \item Pre-deployment calibration conducted indoors
  \item Tynys device co-located with Qingping Air Monitor Gen 2
  \item One-hour synchronized measurement session
  \item Baseline offsets minimized
\end{itemize}
\end{frame}

% Reference Device
\begin{frame}{Reference Device Validation}
\begin{itemize}
  \item Qingping monitor evaluated by SCAQMD
  \item Compared against FEM-grade instruments
  \item PM$_{2.5}$ correlation: R$^2$ = 0.89 -- 0.95
  \item High data recovery and low error
\end{itemize}
\end{frame}

% Mobile Validation
\begin{frame}{In-Cabin Validation}
\begin{itemize}
  \item Both devices installed inside a passenger cabin
  \item Measurements recorded simultaneously
  \item PM, CO$_2$, temperature, and humidity trends closely matched
\end{itemize}
\end{frame}

% Web Platform
\begin{frame}{Web-Based Dashboard}
\begin{itemize}
  \item Live visualization of sensor data
  \item Real-time air quality classification
  \item User-friendly interface for monitoring
\end{itemize}
\end{frame}

% Results Summary
\begin{frame}{Key Outcomes}
\begin{itemize}
  \item Reliable low-cost sensing performance
  \item Strong agreement with reference monitor
  \item Machine learning feasible for automatic classification
  \item Suitable for mobile and in-cabin environments
\end{itemize}
\end{frame}

% Future Work
\begin{frame}{Future Improvements}
\begin{itemize}
  \item Larger dataset collection
  \item Long-term sensor drift analysis
  \item Advanced ML models
  \item Integration with alerts and recommendations
\end{itemize}
\end{frame}

% Closing Slide
\begin{frame}{Conclusion}
\centering
Low-cost IoT + Machine Learning\\
\bigskip
A practical solution for smart air quality monitoring
\end{frame}

\end{document}
